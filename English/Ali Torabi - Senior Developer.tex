%%%%%%%%%%%%%%%%%%%%%%%%%%%%%%%%%%%%%%%%%
% Developer CV
% LaTeX Template
% Version 1.0 (28/1/19)
%
% This template originates from:
% http://www.LaTeXTemplates.com
%
% Authors:
% Jan Vorisek (jan@vorisek.me)
% Based on a template by Jan Küster (info@jankuester.com)
% Modified for LaTeX Templates by Vel (vel@LaTeXTemplates.com)
%
% License:
% The MIT License (see included LICENSE file)
%
%%%%%%%%%%%%%%%%%%%%%%%%%%%%%%%%%%%%%%%%%

%----------------------------------------------------------------------------------------
%	PACKAGES AND OTHER DOCUMENT CONFIGURATIONS
%----------------------------------------------------------------------------------------

\documentclass[a4paper]{developercv} % Default font size, values from 8-12pt are recommended

\usepackage{enumitem}

%----------------------------------------------------------------------------------------

\begin{document}

%----------------------------------------------------------------------------------------
%	TITLE AND CONTACT INFORMATION
%----------------------------------------------------------------------------------------

\begin{minipage}[t]{0.5\textwidth}
	\vspace{-\baselineskip} % Required for vertically aligning minipages

	\colorbox{black}{{\HUGE\textcolor{white}{\textbf{\MakeUppercase{Ali Torabi}}}}}
	\\
	\vspace{6pt}
	\\
	{\huge Senior .NET Developer} % Career or current job title
\end{minipage}
\begin{minipage}[t]{0.30\textwidth} % 27.5% of the page width for the first row of icons
	\vspace{-\baselineskip} % Required for vertically aligning minipages

	% The first parameter is the FontAwesome icon name, the second is the box size and the third is the text
	% Other icons can be found by referring to fontawesome.pdf (supplied with the template) and using the word after \fa in the command for the icon you want
	\icon{MapMarker}{12}{Tehran, Iran}\\
	\icon{Phone}{12}{\href{tel:+989137360539}{+98 913 737 0539}}\\
	\icon{At}{12}{\href{mailto:alitorabi2020@gmail.com}{\small alitorabi2020@gmail.com}}\\
\end{minipage}
\begin{minipage}[t]{0.25\textwidth} % 27.5% of the page width for the second row of icons
	\vspace{-\baselineskip} % Required for vertically aligning minipages

	% The first parameter is the FontAwesome icon name, the second is the box size and the third is the text
	% Other icons can be found by referring to fontawesome.pdf (supplied with the template) and using the word after \fa in the command for the icon you want
	\icon{Linkedin}{12}{\href{https://linkedin.com/in/torabi-ali}{@torabi-ali}}\\
	\icon{Github}{12}{\href{https://github.com/torabi-ali}{@torabi-ali}}\\
	\icon{PaperPlane}{12}{\href{https://t.me/torabi\_ali}{@torabi\_ali}}\\
\end{minipage}

\vspace{0.5cm}

%----------------------------------------------------------------------------------------
%	INTRODUCTION, SKILLS AND TECHNOLOGIES
%----------------------------------------------------------------------------------------

\begin{minipage}[t]{0.5\textwidth} % 35% of the page width for the introduction text
	\cvsect{Who Am I?}
	\vspace{-\baselineskip} % Required for vertically aligning minipages
	\\\par
	Experienced .NET programmer with a demonstrated history of working in the computer software industry.
	I have worked in more than 10 different projects and have a proven track record of delivering high quality software.
	I was also a key member of the team that spearheaded the development of the first version of the application.
\end{minipage}
\hfill % Whitespace between
\begin{minipage}[t]{0.4\textwidth} % 50% of the page for the skills bar chart
	\cvsect{Skills}
	\vspace{-\baselineskip} % Required for vertically aligning minipages
	\begin{itemize}[noitemsep]
		\item C\#
		\item ASP.NET Core \& ASP.NET MVC
		\item Microsoft SQL Server \& PostgreSQL
		\item Redis \& ElasticSearch
		\item Restful API
		\item Git/TFS
		\item Docker
		\item C/C++ \& Rust
	\end{itemize}
\end{minipage}

%----------------------------------------------------------------------------------------
%	EXPERIENCE
%----------------------------------------------------------------------------------------

\cvsect{Experience}
\begin{entrylist}
	\entry
	{2020 -- Current}
	{Senior Full Stack Developer}
	{\href{https://gooshishop.com}{Gooshishop - Hasin Technology.}}
	{I am a senior software engineer working on a large legacy E-Commerce solution. The main challenges are:
	\begin{itemize}[noitemsep]
		\item Maintaining project and fixing numerous bugs and issues.
		\item Optimizing many actions and easing the sales process.
		\item Choosing the best open source E-Commerce solution and implementing our needs in that.
		\item Migrating the whole system to new product.
	\end{itemize}
	I'm proud to say I have improved some actions up to 60\% and Added so many features in these years.
		\\\texttt{ASP.NET MVC}\slashsep\texttt{ASP.NET Core}\slashsep\texttt{SQL Server}\slashsep\texttt{PostgreSQL}\slashsep\texttt{ElasticSearch}}
	\entry
	{2020 -- 2020}
	{Part-Time Senior Full Stack Developer}
	{\href{https://itelemed.ca}{iTelemed - Canada Telemedicine Group Inc.}}
	{I was a part-time senior full stack developer on a fully online medical solution. Main challenges were:
	\begin{itemize}[noitemsep]
		\item Developing the project as fast as possible because of the covid situation.
		\item Developing a lot of features that were not available at that time.
		\item Reaching an architecture to enable us to scale the project.
	\end{itemize}
	Online visiting (video chat), text chat, ticketing and managing patients were some of features built in a 3 month work.
		\\\texttt{ASP.NET Core}\slashsep\texttt{SQL Server}\slashsep\texttt{SignalR}}
	\entry
	{2019 -- 2021}
	{Senior Back End Developer}
	{\href{https://apdp.ir}{Algorithm Pars.}}
	{I was a senior developer working on a legacy project which was widely in use in large companies. The main challenges were:
	\begin{itemize}[noitemsep]
		\item Keeping a legacy software as update as possible to keep the project running smoothly.
		\item Minifying the cost of maintaining the project.
		\item Creating a transition phase in order to migrate to .Net Core
		\item Designing and developing a new framework to migrate from Web Form to ASP.NET Core.
	\end{itemize}
	I was a key person in this project and had a good collaboration with project manager. I had reduced more that 20\% of the line of code and we have improved the performance more than 15\% through years.
		\\\texttt{ASP.NET Core}\slashsep\texttt{ASP.NET WebForms}\slashsep\texttt{SQL Server}}
	\entry
	{2018 -- 2019}
	{Software Consultant \& SEO Manager}
	{Sunseir Travel Agency}
	{Working with my nice colleagues on a website that implements both B2B and B2C business models which handles a very high amount of request. Also improving "Sunseir" brand, advertising and SEO on aside.
		\\\texttt{ASP.NET Core}\slashsep\texttt{SEO Tools}}
	\entry
	{2017 -- 2019\\\footnotesize{Freelance}}
	{.Net Developer}
	{Dipeh.ir}
	{Dipeh is a website to explore about books, authors and publishers. The main goal in this project was designing a software with very specified requirements Another important goal was reaching nice performance through maximum code optimization.
		\\\texttt{ASP.NET MVC}\slashsep\texttt{SQL Server}\slashsep\texttt{Bootstrap}}
	\entry
	{2016 -- 2018}
	{Junior Developer \& SEO Manager}
	{Mehr -o- Mah Travel Agency}
	{This was my first official job as developer. It was a large-scale app with lot's of functionalities that brightened me to professional programming.
		\\\texttt{ASP.NET MVC}\slashsep\texttt{SQL Server}\slashsep\texttt{Bootstrap}\slashsep\texttt{SEO Tools}}
\end{entrylist}

%----------------------------------------------------------------------------------------
%	EDUCATION
%----------------------------------------------------------------------------------------

\cvsect{Education}
\begin{entrylist}
	\entry
	{2016 -- 2018}
	{Master's Degree}
	{Islamic Azad University, Najafabad Branch}
	{I focused on software engineering and software development during MS and therefore my thesis was about predicting software bugs and defects early during development}
	\entry
	{2012 -- 2016}
	{Bachelor's Degree}
	{Zanjan University}
	{}
\end{entrylist}

%----------------------------------------------------------------------------------------
%	ADDITIONAL INFORMATION
%----------------------------------------------------------------------------------------

\cvsect{Languages}
\begin{itemize}
	\setlength\itemsep{3px}
	\item \textbf{Persian} - Native
	\item \textbf{English} - Proficient
\end{itemize}

\end{document}